%\documentclass[10pt]{beamer}
%\documentclass[10pt]{article}
%\usepackage{xcolour-names}
\usepackage[normalem]{ulem}
\errorcontextlines=10
\newtheorem{defi}{Definition}
\mode<presentation>
{
  \usetheme{Warsaw}
  %\usetheme[hideothersubsections]{Goettingen}
  % or ...

  \setbeamercovered{transparent}
  % or whatever (possibly just delete it)

  \subject{Introduction to Machine Learning}
%  \setbeamertemplate{footline}%
%  {%
%    \par\vspace*{-\baselineskip}\par%
%    \raisebox{0.5ex}{\usebeamercolor[fg]{structure}%
%      \ slide \insertframenumber/\inserttotalframenumber}%
%  }%
}
\mode<article>
{
  \usepackage[a4paper,margin=10mm,includehead,headheight=15pt,headsep=5mm]%
  {geometry}
  \usepackage{graphicx}
  \usepackage[noxcolor]{beamerarticle}
  \usepackage{fancyhdr}%
  \usepackage[breaklinks]{hyperref}%
  \renewcommand{\floatpagefraction}{0.75} % default is .5, to increase
  % density.
  \renewcommand*{\bottomfraction}{0.6} % default is 0.3
  \renewcommand*{\topfraction}{0.85} % default is 0.7
  \renewcommand*{\textfraction}{0.1} % default is 0.2
  \pagestyle{fancy}
  \fancyhf{}
  \newlength{\markWidth}
  \setlength{\markWidth}{0.5\textwidth}
  \newlength{\topicnumwidth}
  \settowidth{\topicnumwidth}{1.114.3}
  \addtolength{\markWidth}{-0.6\topicnumwidth}
  %\addtolength{\headheight}{1.92ex}

  \renewcommand{\sectionmark}[1]{\markright{\thesection. #1}}
  \lhead{\parbox[t]{\markWidth}{\raggedright\nouppercase{\rightmark}}}
  \rhead{\thepage}
  \AtBeginDocument{\thispagestyle{empty}}
}


\usepackage[english]{babel}
\usepackage{alltt,booktabs,array,multicol}

%\usepackage[utf8]{inputenc}
% or whatever

%\usepackage{times}
%\usepackage[T1]{fontenc}
% Or whatever. Note that the encoding and the font should match. If T1
% does not look nice, try deleting the line with the fontenc.

% See page 379 of the second edition of LaTeX Companion:
\usepackage{pifont}
\newcommand{\thintick}{\ding{'63}}% thinner
\newcommand{\tick}{\ding{'64}}% fatter
\newcommand{\thincross}{\ding{'67}}% thinner
\newcommand{\cross}{\ding{'70}}% fatter

\date{\today}
\AtBeginSection[]
{
  \begin{frame}<beamer>
    \frametitle{Outline}
    \footnotesize
%    \begin{multicols}{2}
      \tableofcontents[currentsection,currentsubsection]
%    \end{multicols}
  \end{frame}
}

% If you wish to uncover everything in a step-wise fashion, uncomment
% the following command: 

%\beamerdefaultoverlayspecification{<+->}

\newcounter{program}
%\newcommand*{\program}[1]{\refstepcounter{program}\label{#1}\arabic{program}}
% \newcommand*{\program}[1]{%
%    \refstepcounter{program}\hypertarget{#1}{Program \texttt{#1}}%
% }
%\newcommand*{\program}[1]{\refstepcounter{program}\label{#1}\arabic{program}}
\newcommand*{\program}[1]{%
   \hypertarget{#1}{Program \texttt{#1}}%
}

\newcommand*{\linkto}[1]{\hyperlink{#1}{\texttt{#1}}}
\providecommand*{\bs}{\texttt{\char '134}} % Backslash, no break

\newcommand{\cmd}[1]{%
  \texttt{\$ \textbf{#1 \(\hookleftarrow\)}}
}

\newcommand{\cmdbox}[1]{%
  \mbox{\texttt{\$ \textbf{#1 \(\hookleftarrow\)}}}
}

\newcommand{\rootcmd}[1]{%
  \texttt{\# \textbf{#1 \(\hookleftarrow\)}}
}

\newcommand{\opt}[1]{%
  {\bfseries\texttt{#1}}
}

